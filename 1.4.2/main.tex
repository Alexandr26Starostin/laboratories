\documentclass[a4paper]{article}
\usepackage[utf8]{inputenc}
\usepackage[russian,english]{babel}
\usepackage[T2A]{fontenc}
\usepackage[left=10mm, top=20mm, right=18mm, bottom=15mm, footskip=10mm]{geometry}
\usepackage{indentfirst}
\usepackage{amsmath,amssymb}
\usepackage[italicdiff]{physics}
\usepackage{graphicx}
\graphicspath{{images/}}
\DeclareGraphicsExtensions{.pdf,.png,.jpg}
\usepackage{wrapfig}
\usepackage{pgfplots}

\usepackage{caption}
\captionsetup[figure]{name=Рисунок}
\captionsetup[table]{name=Таблица}


\title{\underline{Лабораторная работы 1.4.2}}
\author{Старостин Александр, Б01-401}
\date {9 Ноября, 2024 год}


\begin{document}

\maketitle
\newpage

\textbf{Определение ускорения свободного падения при помощи оборотного маятника}

\section{Аннотация}
    \par \textbf{Цель работы:} определить величину ускорения свободного падения, пользуясь оборотным маятником.\\

    \par \textbf{В работе используются:} оборотный маятник, счётчик числа колебаний, секундамер, штангенциркуль с пределом измерений 1 м.

\section{Теоретические сведения}

Период колебаний физического маятника $T$ с моментом инерции относительность ос качения $I$, массой $m$ и расстоянием от центра масс до оси качения $a$:

\begin{equation}
	T = 2\pi\sqrt{\frac{I}{mga}}
\end{equation}

Пусть положение грузов на оборотном маятнике такого, что периоды колебаний маятника $T_1$ и $T_2$ на призмах $\text{П}_1$ и $\text{П}_2$ совпадают:

\begin{equation}
	T_1 = T_2 = T = 2\pi\sqrt{\frac{I_1}{mgl_1}} = 2\pi\sqrt{\frac{I_2}{mgl_2}}
\end{equation}

По теореме Гюгенса-Штейнера:

\begin{equation}
	I_1 = I_0 + m{l_1}^2,   I_2 = I_0 + m{l_2}^2
\end{equation}

Из (2) и (3), исключая $I_0$ и $m$, получим ($L$ - расстояние между двумя призмами):

\begin{equation}
	g = (l_1 + l_2) \frac{4\pi^2}{T^2} = 4\pi^2 \frac{L}{T^2}
\end{equation}

Это справедливо при различных $l_1$ и $l_2$.\\

Тк не существует точного равенства $T_1$ = $T_2$, то:

\begin{equation}
	T_1 = 2\pi\sqrt{\frac{I_0 + m{l_1}^2}{mgl_1}},  T_2= 2\pi\sqrt{\frac{I_0 + m{l_2}^2}{mgl_2}}
\end{equation}

Тогда получаем, что:

\begin{equation}
	g = 4\pi^2 \frac{L}{{T_0}^2}
\end{equation}

где:

\begin{equation}
	{T_0}^2 = \frac{l_1 {T_1}^2 - l_2 {T_2}^2}{l_1 - l_2}
\end{equation}

Тогда погрешность ускорения свободного падения будет вычислена по формуле:

\begin{equation}
	\varepsilon_g = \sqrt{{\varepsilon_L}^2 + 4{\varepsilon_{T_0}}^2}
\end{equation}

где

\begin{equation}
	\varepsilon_{T_0} = \frac{\sqrt{{l_1}^2 + {l_2}^2}}{l_1 - l_2} \sigma_T \frac{1}{T_0}
\end{equation}

Для того, чтобы измерения и вычисления были точными должно быть верным, что:

\begin{equation}
	1,5 < \frac{l_1}{l_2} < 3
\end{equation}

\section{Ход работы}

\subsection{Измерение масс частей установки}

Был снят стержень с установки и были измерены его масса вместе с грузами и призмами, масса стержня отдельно, массы грузов и массы призмы. Массы нужны для того, чтобы была возможность теоритически вычислить центр инерции (масс) стержня вместе с грузами и призмами. Результаты имерений привидены в таблице:

\begin{table}[h!]
\centering
\caption{Массы частей установки}
\begin{tabular}{|c|c|}
\hline
Часть установки & Масса, г  \\ \hline
Маятник  & $4021.4\pm{0.3}$  \\ \hline
Стержень & $891.5 \pm{0.3}$  \\ \hline
Груз 1   & $1495.3\pm{0.3}$  \\ \hline
Груз 2   & $1481.0\pm{0.3}$  \\ \hline
Призма 1 & $76.6  \pm{0.3}$  \\ \hline
Призма 2 & $77.2  \pm{0.3}$  \\ \hline
\end{tabular}
\end{table}

\subsection{Установка призм на стержень}

Призмы были установлены на стержень таким образом, что концы (острия) призм смотрели внутрь стержня и расстояния от концов стержней до ближайшего острия призм составляло $24\pm{0.01}$ см. Важно, что груз 1 находился между призмами и груз 2 находися между призмой 2 и ближайшем от неё концом стержня.

\subsection{Измерение расстояния между концами (остриями) призм}

С помощью штангенциркуля было измерено расстояние между концами (остриями) призм. Оно составило $51.40\pm{0.01}$ см.

\subsection{Теоретическое вычисление расстояний от концов призм до центра инерции маятника}

Изменяя положение грузов 1 и 2 на стержне, мы изменяли положение центра инерции маятника. Взяв определённое отношение $\frac{l_1}{l_2}$ такое, чтобы оно удолетворяло неравенству (10), мы можем рассчитать значение $l_1$ и $l_2$ и проверить эти значения на практике, те проверить, определяют ли эти значения центр инерции маятника. Результаты вычислений приведены в таблице:

\begin{table}[h!]
\centering
\caption{Теоритические значения $l_1$ и $l_2$}
\begin{tabular}{|c|c|c|}
\hline
$\frac{l_1}{l_2}$ & $l_1$, см & $l_2$, см  \\ \hline
$1.74$  & $32.64$ & $18.76$  \\ \hline
\end{tabular}
\end{table}

\subsection{Измерение расстояний от концов призм до центра инерции маятника}

Установив маятник на т-образную подставку таким образом, чтобы маятник находился в равновесии, мы измерили расстояния от концов призм до точки равновесия маятника. Результаты измерений приведены в процессе:

\begin{table}[h!]
\centering
\caption{Измеренные значения $l_1$ и $l_2$}
\begin{tabular}{|c|c|c|}
\hline
$l_1$, см & $l_2$, см & $\frac{l_1}{l_2}$  \\ \hline
$32.66\pm{0.01}$  & $18.74\pm{0.01}$ & $1.7428$  \\ \hline
\end{tabular}
\end{table}

\subsection{Измерение периода $T_2$ колебаний маятника, когда сверху груз 2}

Поместив маятник на установку 2-ым грузом вверх, мы измеряли периоды колебаний маятника, чтобы определить средний период колебаний. Амплитуда колебаний составляла $5^{\circ}$. Результаты измерений приведены в таблице:

\begin{table}[h!]
\centering
\caption{Определение среднего периода $\overline{T_2}$}
\begin{tabular}{|c|c|c|}
\hline
$N$ - количество колебаний & $t_2$ - время всех колебний, с & $T_2 = \frac{t_2}{N}$ - период колебаний, с  \\ \hline
$20$  & $29.91\pm{0.03}$ & $1.4955\pm{0.0015}$   \\ \hline
$20$  & $29.90\pm{0.03}$ & $1.4950\pm{0.0015}$   \\ \hline
$20$  & $29.90\pm{0.03}$ & $1.4950\pm{0.0015}$   \\ \hline
$20$  & $29.90\pm{0.03}$ & $1.4950\pm{0.0015}$   \\ \hline
\end{tabular}
\end{table}

\item Погрешность периода: $\sigma_{T_2} = T_2\frac{\sigma_{t_2}}{t_2}$

\item Погрешность вычисления среднего периода: $\sigma_{\overline{T_2}} = \sqrt{\frac{1}{n(n-1)}\sum_{i=1}^{n} ({T_2}_i - \overline{T_2})^2}$

\item Средний период: $\overline{T_2} = 1.4951\pm{0.0001}$ с.

\subsection{Измерение периода $T_1$ колебаний маятника, когда сверху груз 1}

Поместив маятник на установку 1-ым грузом вверх, мы измеряли периоды колебаний маятника, чтобы определить средний период колебаний. Амплитуда колебаний составляла $5^{\circ}$. Результаты измерений приведены в таблице:

\begin{table}[h!]
\centering
\caption{Определение среднего периода $\overline{T_1}$}
\begin{tabular}{|c|c|c|}
\hline
$N$ - количество колебаний & $t_1$ - время всех колебний, с & $T_1 = \frac{t_1}{N}$ - период колебаний, с  \\ \hline
$20$  & $29.43\pm{0.03}$ & $1.4715\pm{0.0015}$   \\ \hline
$20$  & $29.44\pm{0.03}$ & $1.4720\pm{0.0015}$   \\ \hline
$20$  & $29.43\pm{0.03}$ & $1.4715\pm{0.0015}$   \\ \hline
$20$  & $29.43\pm{0.03}$ & $1.4715\pm{0.0015}$   \\ \hline
\end{tabular}
\end{table}

\item Погрешность периода: $\sigma_{T_1} = T_1\frac{\sigma_{t_1}}{t_1}$

\item Погрешность вычисления среднего периода: $\sigma_{\overline{T_1}} = \sqrt{\frac{1}{n(n-1)}\sum_{i=1}^{n} ({T_1}_i - \overline{T_1})^2}$

\item Средний период: $\overline{T_1} = 1.4716\pm{0.0001}$ с.

\subsection{Оценка различия значений $\overline{T_1}$ и $\overline{T_2}$}

Из-за неидеальности условий проведения измерений $\overline{T_1}$ и $\overline{T_2}$ не совпадают. Оценим, на сколько они различаются:

\item $\frac{\Delta T}{T} = \frac{\overline{T_2} - \overline{T_1}}{\overline{T_2}} = 0.0157 = 1.57 \% $

Различие между $\overline{T_1}$ и $\overline{T_2}$ незначительное, значит измеренные периоды подходят для определения ускорения свободного падения.

\subsection{Измерения для определения ускорения свободного падения}

Чтобы определить ускорения свободного падения, можно повторить действия из пунктов (5) - (9), посторить линейный график и определить коэффициент наклона графика, а по нему и ускорение свободного падения. \\

Из формул (6) и (7) следует:

\begin{equation}
	({T_1}^2 l_1 - {T_2}^2 l_2) \frac{g}{4{\pi}^2} = {l_1}^2 - {l_2}^2
\end{equation}

Формула (11) является линейной зависимостью, по которой из углового коэффициента можно найти $g$.\\

Измерения нужных нам величин для построения графика приведены в таблице:

\newpage

\begin{table}[h!]
\centering
\caption{Величины для построения графика}
\begin{tabular}{|c|c|c|c|c|}
\hline
Номер измерения & $T_1$, с & $T_2$, с & $l_1$, см & $l_2$, см \\ \hline
1   & $1.4716\pm{0.0001}$ & $1.4951\pm{0.0001}$ & $32.66\pm{0.01}$ & $18.74\pm{0.01}$ \\ \hline
2   & $1.4812\pm{0.0001}$ & $1.5116\pm{0.0001}$ & $31.10\pm{0.01}$ & $18.30\pm{0.01}$ \\ \hline
3   & $1.4907\pm{0.0001}$ & $1.5507\pm{0.0001}$ & $33.86\pm{0.01}$ & $17.54\pm{0.01}$ \\ \hline
4   & $1.5050\pm{0.0001}$ & $1.5682\pm{0.0001}$ & $32.81\pm{0.01}$ & $18.59\pm{0.01}$ \\ \hline
5   & $1.5375\pm{0.0001}$ & $1.6241\pm{0.0001}$ & $34.66\pm{0.01}$ & $16.74\pm{0.01}$ \\ \hline
6   & $1.5348\pm{0.0001}$ & $1.6404\pm{0.0001}$ & $33.65\pm{0.01}$ & $17.75\pm{0.01}$ \\ \hline
7   & $1.5257\pm{0.0001}$ & $1.6173\pm{0.0001}$ & $34.48\pm{0.01}$ & $16.92\pm{0.01}$ \\ \hline
8   & $1.5433\pm{0.0001}$ & $1.6431\pm{0.0001}$ & $34.73\pm{0.01}$ & $16.67\pm{0.01}$ \\ \hline
9   & $1.4898\pm{0.0001}$ & $1.5095\pm{0.0001}$ & $31.64\pm{0.01}$ & $19.76\pm{0.01}$ \\ \hline
10  & $1.4844\pm{0.0001}$ & $1.5115\pm{0.0001}$ & $31.97\pm{0.01}$ & $19.43\pm{0.01}$ \\ \hline
\end{tabular}
\end{table}

\subsection{Построение графика для определения по нему ускорения свободного падения}

По данным из пункта 3.10 построим график для зависимости из формулы (11):

\begin{figure}[!h]
\centering
\begin{tikzpicture}
\begin{axis}
[
name = plot,
ylabel = {${l_1}^2 - {l_2}^2$, см$^2$},
xlabel = {$({T_1}^2 l_1 - {T_2}^2 l_2)$, см*с$^2$},
width = 450,
height = 450,
]


\addplot[green, mark = *] coordinates
{
(28.83874, 715.488)
};

\addplot[green, mark = *] coordinates
{
(26.41765, 632.32)
};
\addplot[green, mark = *] coordinates
{
(33.06531, 838.848)
};
\addplot[green, mark = *] coordinates
{
(28.59799, 730.908)
};
\addplot[green, mark = *] coordinates
{
(37.77788, 921.088)
};
\addplot[green, mark = *] coordinates
{
(31.50262, 817.260)
};
\addplot[green, mark = *] coordinates
{
(36.00423, 902.284)
};
\addplot[green, mark = *] coordinates
{
(37.71375, 928.284)
};
\addplot[green, mark = *] coordinates
{
(25.20016, 610.632)
};
\addplot[green, mark = *] coordinates
{
(26.05368, 644.556)
};
\addplot[green, mark = *] coordinates
{
(24, 594.60068)
(39, 973.11173)
}; \label {stall}


\end{axis}
\node[anchor = north west] (legend) at (plot.north west)
{\begin{tabular}{l l}
Наилучшая прямая для зависимости из формулы (11) & \ref{stall}\\
\end{tabular}};

\end{tikzpicture}
\caption{Линейная зависимость из формулы (11)}
\end{figure}

\item По МНК построим наилучшую прямую $y = kx + b$ для зависмости из формулы (11):

\item $k = \frac{<xy> - <x><y>}{<x^2> - <x>^2} = 25.23407$, см/с$^2$
\item $b = <y> - k<x> = -11.017$, см$^2$

\item $\sigma_k = \frac{1}{\sqrt{n}} \sqrt{\frac{<y^2> - <y>^2}{<x^2> - <x>^2} - k^2} = 1.22299$, см/с$^2$

\subsection{Результат измерения ускорения свободного падения}

Из коэффициента угла наклона прямой найдём $g$:

\item $g = 4\pi^2 k = 995.192$ см/с$^2$  $= 9.95192$ м/с$^2$
\item $\sigma_g = g \frac{\sigma_k}{k} = 48.233$ см/с$^2$ $= 0.48233$ м/с$^2$

\item Тогда ускорение свободного падения составляет: $g = 9.95192\pm{0.48233}$ м/с$^2$

\section{Вывод}

Мы измерили величину ускорения свободного падения $g = 9.95192\pm{0.48233}$ м/с$^2$, используя оборотный маятник.

\end{document}
