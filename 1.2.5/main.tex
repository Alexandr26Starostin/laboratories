\documentclass[a4paper]{article}
\usepackage[utf8]{inputenc}
\usepackage[russian,english]{babel}
\usepackage[T2A]{fontenc}
\usepackage[left=10mm, top=20mm, right=18mm, bottom=15mm, footskip=10mm]{geometry}
\usepackage{indentfirst}
\usepackage{amsmath,amssymb}
\usepackage[italicdiff]{physics}
\usepackage{graphicx}
\graphicspath{{images/}}
\DeclareGraphicsExtensions{.pdf,.png,.jpg}
\usepackage{wrapfig}
\usepackage{pgfplots}

\usepackage{caption}
\captionsetup[figure]{name=Рисунок}
\captionsetup[table]{name=Таблица}


\title{\underline{Лабораторная работы 1.2.5}}
\author{Старостин Александр, Б01-401}
\date {9 Ноября, 2024 год}


\begin{document}

\maketitle
\newpage

\textbf{Исследование вынужденной регулярной прицессии гироскопа}

\section{Аннотация}
    \par \textbf{Цель работы:} исследовать вынужденную прецессию гироскопа, установить зависимость скорости вынужденной прецессии от величины момента сил, действующий на ось гироскопа и сравнить ее со скоростью, рассчитанной по скорости прецессии.\\

    \par \textbf{В работе используются:} гироскоп в кардановом подвесе, секундомер, набор грузов, отдельный ротор гироскопа, цилиндр известной массы, крутильный маятник, штангенсциркуль, линейка.


\section{Теоретические сведения}


Из основного соотношения гироскопии можно выразить уголовую скорость вращения прецессии $\Omega$ от массы груза на гироскопе $m$, расстояния от центра гироскопа до груза $l$, момента инерции гироскопа относительно главной оси, вокруг которой происходит вращение, $I_\text{р}$ и угловой скорости вращения гироскопа $\omega_0$:

\begin{equation}
	\Omega = \frac{mgl}{I_\text{р}\omega_0}
\end{equation}

Момент инерции ротора относительно оси симметрии $I_0$ измеряется по крутильным колебаниям точной копии ротора, подвешиваемой вдоль оси симметрии на десткой проволоке. Период крутильных колебаний $T_0$ зависит от момента инерции $I_0$ и модуля кручения проволоки $f$:

\begin{equation}
	T_0 = 2\pi\sqrt{\frac{I_0}{f}}
\end{equation}

Чтобы исключить модуль кручения проволоки, вместо ротора гироскопа к той же проволоке подвешивают цилиндр правильной формы с известными размерами и массой, для которого можно вычислить момент инерции $I_\text{ц}$. И зная период крутильных колебаний ротора и цилиндра на одной и той же проволке, мы сможем найти $I_\text{р}$ из соотношения:

\begin{equation}
	\frac{T_\text{ц}^2}{I_\text{ц}} = \frac{T_\text{р}^2}{I_\text{р}}
\end{equation}

Из формулы 1 мы сможем найти момент импульса ротора $L$, предварительно измерив уголовую скорость вращения прецессии $\Omega$:

\begin{equation}
	L = I_\text{р}\omega_0 = \frac{mgl}{\Omega}
\end{equation}

Из-за сил трения, возникающих в креплениях гироскопа при вращении вокруг оси прецессии, возникает сил трения, из-за которого ось основного вращения гироскопа падает. Момент сил трения $M_\text{тр}$ можно вычислить по формуле:

\begin{equation}
	M_\text{тр} = \Omega_\text{тр} L
\end{equation}

Из формулы (4) можно можно выразить угловую скорость вращения гироскопа $\omega_0$:

\begin{equation}
	\omega_0 = \frac{mgl}{\Omega I_\text{р}}
\end{equation}

Так же углофивую скорость вращения гироскопа $\omega_0$ можно вычислить графичским методом. Известно, что фигуры Лиссажу на осцилографе возникают, если совпадают частоты на каналах X и Y. Подадим на один из них частоту вращения гироскопа (её мы не знаем и хотим измерить), а на другой - выбранную нами частоту на генераторе частот. Отключив гироскоп от питания, мы получим, что из-за сил трения уголовая частота вращения гироскопа $\omega$ будет падать с приблизительно постоянным угловым ускорением $\varepsilon$. Посторив линейный график зависимости $\omega(t)$ и найдя по нему $\omega(0)$, мы получим $\omega_0$.

\section{Ход работы}

\subsection{Установка оси гироскопа в горизонтальное положение}

Установим ось гироскопа в горизонтальное положение, осторожно поворачивая его. Так же нужно снять все предметы, подвешенные на гироскоп, чтобы во время разгона гироскопа не возинкала прецессия оси вращения ротора (в нашем случае этим предметом была гайка).

\subsection{Запуск гироскопа}

Включим питание гироскопа и подождём 4-5 минут, чтобы вращение ротора успело стабилизироваться.

\subsection{Проверка устойчивости оси гироскопа}

Легко постукивая ось вращения ротора, мы убедились, что ось не изменяет своего положения в пространстве. Надавливая на ось вращения ротора гироскопа, наблюдая за направлением прецесси оси и пользуясь векторной формой основного соотношения гироскопии, мы могли определить направления угловой скорости вращения гироскопа $\omega_0$ и импульса гироскопа относительно оси вращения ротора $L$. \\

Векторная форма основного соотношения гироскопии:

\begin{equation}
	\Vec{M} = [\Vec{\Omega} \Vec{L}]
\end{equation}

\subsection{Проверка наличия прецессии оси вращения ротора}

Подвесив груз к оси вращения ротора гироскопа, мы наблюдали прецессию этой оси, возникшую под действием силы тяжести груза (горизонтальная прецессия). Тк из-за прецессии возникло вращение гироскопа в его крплениях, то в них возникли силы трения. Из-за чего возник момент сил трения относительно оси вращения ротора гироскопа, что привело к ещё одной прецессии оси (вертикальная прецессия).

\subsection{Измерение прецессий оси ротора и моментов силы тяжести}

Измерим горизонтальную и вертикальную прецессии оси ротора гироскопа и моменты сил тяжести. \\

Для измерения горизонтальной прецессии будем засекать время, за  которое гироскоп делает полное число оборотов вокруг вертикали.\\

Для измерения вертикальной прецессии оси ротора гироскопа будем засекать время за которое носик гироскопа опустится на $\approx 10^\circ$. Для этого в начале опыта отклоним ось ротора гироскопа на $\approx 5^\circ$ так, чтобы грузик был выше горизонтали. Во время вращения гироскопа будем измерять угол между грузиком и горизонталью до тех пор, пока он не составит $\approx -5^\circ$, те грузик будет ниже горизонтали. На этом моменте мы закончим засекать время падения оси ротора гироскопа. \\

Чтобы определять углы, мы с помощью треугольника можем измерить до начала вращения гироскопа высоты кончика оси вращения ротора относительно подставки горископа, которые будут соответствовать отклонению кончика оси на $5^\circ$ и $-5^\circ$. Тогда для во вращения нам нужно будет с контролировать только высоту кончика оси ротора. Данные измерения приведены в таблице:

\begin{table}[h!]
\centering
\caption{Соответсвие высот кончика оси и углов отклоения оси}
\begin{tabular}{|c|c|c|c|}
\hline
Расстояние от центра кончика $l$, $\text{мм}$ & Угол отклонения $\alpha$, $^\circ$ & Начальная высота $h_0$, $\text{мм}$ & Высота отклонения $h$, $\text{мм}$  \\ \hline
$122\pm{10}$  & $5$  & $113\pm{1}$ &  $124\pm{1}$\\ \hline
$122\pm{10}$  & $-5$ & $113\pm{1}$ &  $102\pm{1}$\\ \hline
\end{tabular}
\end{table}

Все величины в таблице связаны соотношением (угол $\alpha$ - малый):

\begin{equation}
	h = h_0 + l\sin{\alpha} \approx h_0 + l\alpha
\end{equation}

Из него получаем:

\begin{equation}
	\alpha = \frac{h - h_0}{l}
\end{equation}

\item Погрешность измерения угла составляет: $\sigma_\alpha = \alpha \sqrt{\frac{\sigma_h^2 + \sigma_{h_0}^2}{(h - h_0)^2} + (\frac{\sigma_l}{l})^2} = 0,01\text{ рад} = 0.76^\circ$\\

Далее в таблице приведены все нужные нам измерения ($m$ - масса груза, $l$ - расстояние от центра до груза/кончика, $t_\text{обр}$ - время обращения по горизонтали, $N$ - количество горизонтальных обращений, $t_\text{пад}$ - время падения по вертикали, $\alpha$ - угол падения, $M$ - момент силы тяжести груза, $T$ - период обращения по горизонтали, $\Omega$ - угловая скорость горизонтальной прецессии, $\Omega_\text{тр}$ - угловая скорость вертикальной прецессии):

\newpage

\begin{table}[h!]
\centering
\caption{Измерение прецессий оси и моментов сил тяжести}
\begin{tabular}{|c|c|c|c|c|c|c|c|c|c|}
\hline
$m\text{, г}$ & $l\text{, мм}$ & $t_\text{обр}\text{, c}$ & $N$ & $t_\text{пад}\text{, c}$ & $\alpha\text{, рад}$ & $M\text{, Н*м}$ & $T\text{, c}$ & $\Omega\text{, c$^{-1}$}$ & $\Omega_\text{тр}\text{, c$^{-1}$}$\\ \hline

$343\pm{0.3}$  & $122\pm{10}$  & $86.2\pm{0.3}$ &  $3$ & $86.2\pm{0.3}$ & $0.17\pm{0.01}$ & $0.41$ & $28.7$ & $0.219$ & $0.00197$\\ \hline

$343\pm{0.3}$  & $122\pm{10}$  & $116.3\pm{0.3}$ &  $4$ & $84.3\pm{0.3}$ & $0.17\pm{0.01}$ & $0.41$ & $29.1$ & $0.216$ & $0.00202$\\ \hline

$273\pm{0.3}$  & $122\pm{10}$  & $108.1\pm{0.3}$ &  $3$ & $86.3\pm{0.3}$ & $0.17\pm{0.01}$ & $0.33$ & $36.0$ & $0.174$ & $0.00197$\\ \hline

$273\pm{0.3}$  & $122\pm{10}$  & $109.8\pm{0.3}$ &  $3$ & $87.5\pm{0.3}$ & $0.17\pm{0.01}$ & $0.33$ & $36.6$ & $0.172$ & $0.00194$\\ \hline

$220\pm{0.3}$  & $122\pm{10}$  & $89.8\pm{0.3}$ &  $3$ & $89.8\pm{0.3}$ & $0.17\pm{0.01}$ & $0.26$ & $29.9$ & $0.210$ & $0.00189$\\ \hline

$220\pm{0.3}$  & $122\pm{10}$  & $134.6\pm{0.3}$ &  $4$ & $90.6\pm{0.3}$ & $0.17\pm{0.01}$ & $0.26$ & $33.7$ & $0.187$ & $0.00188$\\ \hline

$176\pm{0.3}$  & $122\pm{10}$  & $112.1\pm{0.3}$ &  $2$ & $90.1\pm{0.3}$ & $0.17\pm{0.01}$ & $0.21$ & $56.1$ & $0.112$ & $0.00189$\\ \hline

$176\pm{0.3}$  & $122\pm{10}$  & $111.8\pm{0.3}$ &  $2$ & $91.3\pm{0.3}$ & $0.17\pm{0.01}$ & $0.21$ & $55.9$ & $0.112$ & $0.00186$\\ \hline

$141\pm{0.3}$  & $122\pm{10}$  & $137.0\pm{0.3}$ &  $2$ & $96.2\pm{0.3}$ & $0.17\pm{0.01}$ & $0.17$ & $68.5$ & $0.092$ & $0.00177$\\ \hline

$141\pm{0.3}$  & $122\pm{10}$  & $138.8\pm{0.3}$ &  $2$ & $98.2\pm{0.3}$ & $0.17\pm{0.01}$ & $0.17$ & $69.4$ & $0.090$ & $0.00173$\\ \hline

$116\pm{0.3}$  & $122\pm{10}$  & $167.6\pm{0.3}$ &  $2$ & $101.2\pm{0.3}$ & $0.17\pm{0.01}$ & $0.14$ & $83.8$ & $0.075$ & $0.00168$\\ \hline

$116\pm{0.3}$  & $122\pm{10}$  & $167.5\pm{0.3}$ &  $2$ & $102.7\pm{0.3}$ & $0.17\pm{0.01}$ & $0.14$ & $83.8$ & $0.075$ & $0.00166$\\ \hline

$93\pm{0.3}$  & $122\pm{10}$  & $103.7\pm{0.3}$ &  $1$ & $104.3\pm{0.3}$ & $0.17\pm{0.01}$ & $0.11$ & $103.7$ & $0.061$ & $0.00163$\\ \hline

$93\pm{0.3}$  & $122\pm{10}$  & $104.4\pm{0.3}$ &  $1$ & $104.4\pm{0.3}$ & $0.17\pm{0.01}$ & $0.11$ & $104.4$ & $0.061$ & $0.00163$\\ \hline

$77\pm{0.3}$  & $122\pm{10}$  & $123.2\pm{0.3}$ &  $1$ & $106.1\pm{0.3}$ & $0.17\pm{0.01}$ & $0.09$ & $123.2$ & $0.051$ & $0.00160$\\ \hline

$77\pm{0.3}$  & $122\pm{10}$  & $125.6\pm{0.3}$ &  $1$ & $108.7\pm{0.3}$ & $0.17\pm{0.01}$ & $0.09$ & $125.6$ & $0.050$ & $0.00156$\\ \hline

$57\pm{0.3}$  & $122\pm{10}$  & $165.8\pm{0.3}$ &  $1$ & $110.2\pm{0.3}$ & $0.17\pm{0.01}$ & $0.07$ & $165.8$ & $0.038$ & $0.00154$\\ \hline

$57\pm{0.3}$  & $122\pm{10}$  & $164.6\pm{0.3}$ &  $1$ & $111.2\pm{0.3}$ & $0.17\pm{0.01}$ & $0.07$ & $164.6$ & $0.038$ & $0.00153$\\ \hline

\end{tabular}
\end{table}


Формулы связей между величинами таблицы:

\begin{equation}
	M = mgl
\end{equation}

\begin{equation}
	T = \frac{t_\text{обр}}{N}
\end{equation}

\begin{equation}
	\Omega = \frac{2\pi}{T}
\end{equation}

\begin{equation}
	\Omega_\text{тр} = \frac{\alpha}{t_\text{пад}}
\end{equation}

\item $\varepsilon_M = \sqrt{(\frac{\sigma_m}{m})^2 + (\frac{\sigma_l}{l})^2} = 0.082 = 8.2 \%$ \\

\item $\varepsilon_T = \frac{\sigma_{t_\text{обр}}}{t_\text{обр}} = 0.0024 = 0.24 \%$ \\

\item $\varepsilon_\Omega = \varepsilon_T = 0.0024 = 0.24 \%$ \\

\item $\varepsilon_{\Omega_\text{тр}} = \sqrt{(\frac{\sigma_\alpha}{\alpha})^2 + (\frac{\sigma_{t_\text{пад}}}{t_\text{пад}})^2} = 0.059 = 5.9 \%$ \\


\subsection{Построение графика зависимости $M$ от $\Omega$}

Посторим график зависимости $M$ от $\Omega$

\newpage

\begin{figure}[!h]
\centering
\begin{tikzpicture}
\begin{axis}
[
name = plot,
ylabel = {$M\text{, Н*м}$},
xlabel = {$\Omega\text{, c$^{-1}$}$},
xmin = 0,
ymin = 0,
width = 450,
height = 450,
]


\addplot[green, mark = *] coordinates
{
(0.219, 0.410)
};

\addplot[green, mark = *] coordinates
{
(0.174, 0.326)
};
\addplot[green, mark = *] coordinates
{
(0.210, 0.333)
};
\addplot[green, mark = *] coordinates
{
(0.112, 0.210)
};
\addplot[green, mark = *] coordinates
{
(0.092, 0.145)
};
\addplot[green, mark = *] coordinates
{
(0.075, 0.139)
};
\addplot[green, mark = *] coordinates
{
(0.061, 0.111)
};
\addplot[green, mark = *] coordinates
{
(0.051, 0.092)
};
\addplot[green, mark = *] coordinates
{
(0.038, 0.068)
};






\addplot[green, mark = *] coordinates
{
(0.216, 0.410)
};

\addplot[green, mark = *] coordinates
{
(0.172, 0.326)
};
\addplot[green, mark = *] coordinates
{
(0.187, 0.293)
};
\addplot[green, mark = *] coordinates
{
(0.112, 0.210)
};
\addplot[green, mark = *] coordinates
{
(0.090, 0.169)
};
\addplot[green, mark = *] coordinates
{
(0.075, 0.139)
};
\addplot[green, mark = *] coordinates
{
(0.060, 0.111)
};
\addplot[green, mark = *] coordinates
{
(0.050, 0.092)
};
\addplot[green, mark = *] coordinates
{
(0.038, 0.068)
};


\addplot[green, mark = *] coordinates
{
(0, 0.0138)
(0.230, 0.39077)
}; \label {stall}


\end{axis}
\node[anchor = north west] (legend) at (plot.north west)
{\begin{tabular}{l l}
Наилучшая прямая для зависимости $M = L\Omega$ & \ref{stall}\\
\end{tabular}};

\end{tikzpicture}
\caption{Зависимость $M$ от $\Omega$}
\end{figure}

\item График зависмости линейный, значит основное соотношение гироскопии верно.\\
\item По МНК построим наилучшую прямую $y = kx + b$ для зависмости $M = L\Omega$ ($k = L$):

\item $L = k = \frac{<xy> - <x><y>}{<x^2> - <x>^2} = 1.639 \text{ м$^2$кг/с}$
\item $b = <y> - k<x> = 0.0138 \text{, Н*м}$

\item $\sigma_L = \sigma_k = \frac{1}{\sqrt{n}} \sqrt{\frac{<y^2> - <y>^2}{<x^2> - <x>^2} - k^2} = 0.117 \text{ м$^2$кг/с}$

\\

\item Тогда $L = 1.639 \pm{0.117} \text{ м$^2$кг/с}$

\subsection{Измерение момента инерции ротора}

Из формулы (3) получаем, что ($M_\text{ц}$ - масса цилиндра, $d$ - диаметр цилиндра, $T_\text{р}$ - период крутильных колебаний ротора, $T_\text{ц}$ - период крутильных колебаний цилиндра):

\begin{equation}
	 I_\text{р} = I_\text{ц} \frac{T_\text{р}^2}{T_\text{ц}^2} = \frac{1}{8} M_\text{ц}d^2 \frac{T_\text{р}^2}{T_\text{ц}^2}
\end{equation}

\newpage

Измерим все нужные нам величины:

\begin{table}[h!]
\centering
\caption{Данные для измерения моментов инерции тел}
\begin{tabular}{|c|c|c|c|c|}
\hline
Тело & $m\text{, г}$ & $d\text{, мм}$ & Кол. колебаний $N$ & Время колебаний $t\text{, c}$  \\ \hline

Цилиндр & $1102.2\pm{0.3}$ & $58.1\pm{0.1}$ & $10$ & $40.2\pm{0.3}$  \\ \hline

Цилиндр & $1102.2\pm{0.3}$ & $58.1\pm{0.1}$ & $10$ & $40.9\pm{0.3}$  \\ \hline

Ротор & - & - & $10$ & $31.9\pm{0.3}$ \\ \hline

Ротор & - & - & $10$ & $31.6\pm{0.3}$ \\ \hline

\end{tabular}
\end{table}

Получаем, что:

\item $T_\text{р} = 4.06 \pm{0.04} \text{, c}$
\item $T_\text{ц} = 3.18 \pm{0.02} \text{, c}$\\

Тогда:

\item $I_\text{р} = 0.000758 \text{, м$^2$кг}$

\item $\sigma_{I_\text{р}} = I_\text{р} \sqrt{(\frac{\sigma_{M_\text{ц}}}{M_\text{ц}})^2 + 4 (\frac{\sigma_d}{d})^2 + 4 (\frac{\sigma_{T_\text{ц}}}{T_\text{ц}})^2 + 4 (\frac{\sigma_{T_\text{р}}}{T_\text{р}})^2} = 0.000094 \text{, м$^2$кг}$\\

\\\\

\item Получаем $I_\text{р} = 0.000758 \pm{0.000094} \text{, м$^2$кг}$


\subsection{Измерение частоты вращения ротора}

Момент импульса ротора $L$, момент инерции ротора $I_\text{р}$, угловая скорость вращения ротора $\omega_0$ и частота вращения ротора $\nu$ связаны через соотношение:
\begin{equation}
    L = \omega_0 I_\text{р} = \nu 2\pi I_\text{р}
\end{equation}

Тогда:

\item  $\nu = \frac{\omega_0}{2\pi} = \frac{L}{I_\text{р}} = 344.27 \text{, с$^{-1}$}$
\item $\sigma_{\omega_0} = \omega_0 \sqrt{(\frac{\sigma_L}{L})^2 + (\frac{\sigma_{I_\text{р}}}{I_\text{р}})^2} = 30.65 \text{, с$^2$}$\\

Тогда:

\item $\omega_0 = 344.27 \pm{30.65} \text{, с$^{-1}$}$

\subsection{Измерение момента сил трения}

Из 2 таблицы получаем:

\item $\overline{\Omega_\text{тр}} = 0.00176 \pm{0.00034} \text{, c$^{-1}$}$\\

Из формулы (5) получаем:

\item $M_\text{тр} = 0.00288 \text{, Н*м}$\\

\item $\sigma_{M_\text{тр}} = M_\text{тр} \sqrt{(\frac{\sigma_{\Omega_\text{тр}}}{\Omega_\text{тр}})^2 + (\frac{\sigma_L}{L})^2} = 0.00056 \text{, Н*м}$\\

Тогда:

\item $M_\text{тр} = 0.00288 \pm{0.00056} \text{, Н*м}$\\

\subsection{Графический метод определения начальной угловой скорости ротора}

Графический метод определения начальной угловой скорости ротора описан в теории. \\

\newpage

В таблице приведены значения, полученные с помощью осцилографа, для построения графика $\nu \text{ от } t$:

\begin{table}[h!]
\centering
\caption{Зависимость $\nu \text{ от } t$}
\begin{tabular}{|c|c|}
\hline
Угловая скорость вращения ротора  $\omega\text{, c$^{-1}$}$ & Время $t\text{, c}$\\ \hline

$380\pm{0.3}$  & $37.9\pm{0.3}$  \\ \hline
$360\pm{0.3}$  & $109.0\pm{0.3}$  \\ \hline
$340\pm{0.3}$  & $183.2\pm{0.3}$  \\ \hline
$320\pm{0.3}$  & $261.2\pm{0.3}$  \\ \hline
$300\pm{0.3}$  & $341.8\pm{0.3}$  \\ \hline
$280\pm{0.3}$  & $427.8\pm{0.3}$  \\ \hline
$260\pm{0.3}$  & $517.2\pm{0.3}$  \\ \hline
$240\pm{0.3}$  & $611.1\pm{0.3}$  \\ \hline
$220\pm{0.3}$  & $711.4\pm{0.3}$  \\ \hline
$200\pm{0.3}$  & $818.6\pm{0.3}$  \\ \hline

\end{tabular}
\end{table}

Посторим график зависмости $\nu \text{ от } t$:

\begin{figure}[!h]
\centering
\begin{tikzpicture}
\begin{axis}
[
name = plot,
ylabel = {$\nu\text{, c$^{-1}$}$},
xlabel = {$t$, c},
xmin = 0,
ymin = 0,
width = 450,
height = 450,
]


\addplot[green, mark = *] coordinates
{
(37.93, 380)
};

\addplot[green, mark = *] coordinates
{
(109.03, 360)
};
\addplot[green, mark = *] coordinates
{
(183.16, 340)
};
\addplot[green, mark = *] coordinates
{
(261.21, 320)
};
\addplot[green, mark = *] coordinates
{
(341.79, 300)
};
\addplot[green, mark = *] coordinates
{
(427.83, 280)
};
\addplot[green, mark = *] coordinates
{
(517.15, 260)
};
\addplot[green, mark = *] coordinates
{
(611.12, 240)
};
\addplot[green, mark = *] coordinates
{
(711.42, 220)
};
\addplot[green, mark = *] coordinates
{
(818.55, 200)
};


\addplot[green, mark = *] coordinates
{
(0, 382.79)
(900.00, 174.89)
}; \label {stall}


\end{axis}
\node[anchor = north west] (legend) at (plot.north west)
{\begin{tabular}{l l}
Наилучшая прямая для зависимости $\nu(t)$ & \ref{stall}\\
\end{tabular}};

\end{tikzpicture}
\caption{Зависимость $\nu$ от $t$}
\end{figure}


\item По МНК построим наилучшую прямую $y = kx + b$ для зависмости $\omega(t)$ ($b = \nu_0$ - начальная частота ротора) ($k =  \frac{\varepsilon}{2\pi}$ - угловое ускорение ротора):

\item $k = \frac{<xy> - <x><y>}{<x^2> - <x>^2} = -0.231 \text{, c$^{-2}$}$
\item $\nu_0 = b = <y> - k<x> = 382.79 \text{, c$^{-1}$}$

\item $\sigma_k = \frac{1}{\sqrt{n}} \sqrt{\frac{<y^2> - <y>^2}{<x^2> - <x>^2} - k^2} = 0.004 \text{, c$^{-2}$}$

\item $\sigma_{\nu_0} = \sigma_b = \sigma_k \sqrt{<x^2> - <x>^2} = 0.98 \text{, c$^{-1}$}$\\

\\

\item Тогда $\nu_0 = 382.79 \pm{0.98} \text{ м$^2$кг/с}$

\item Измерим момент сил трения внутри гироскопа, которые тормозят его вращение:

\item $M_\text{внутр} = \varepsilon I_\text{р} = k2\pi I_\text{р} = 0.0011 \text{ Н*м}$

\item $\sigma_{M_\text{внутр}} = M_\text{внутр} \sqrt{(\frac{\sigma_k}{k})^2 + (\frac{\sigma_{I_\text{р}}}{I_\text{р}})^2} = 0.0002 \text{ Н*м}$

\item Тогда $M_\text{внутр} = 0.0011 \pm{0.0002} \text{ Н*м}$

\subsection{Сравнение угловых скоростей прецессии и вращения оси ротора}

Из числовых значений видно, что $\Omega << \omega_0$ и $\Omega_\text{тр} << \omega_0$. Значит условие для выполнения основного соотношения гироскопии верно.

\section{Вывод}

Мы проверили верность основного соотношения гироскопии, имерели момент импульса ротора $L$, момент инерции ротора $I_\text{р}$, момент сил трения $M_\text{тр}$, угловую скорость вращения ротора $\omega_0$ теоретически и графически (результаты измерений близки к друг другу) и проверили, что угловые скорости прецессий много меньше угловой скорости вращения ротора.

\end{document}




