\documentclass[a4paper]{article}
\usepackage[utf8]{inputenc}
\usepackage[russian,english]{babel}
\usepackage[T2A]{fontenc}
\usepackage[left=10mm, top=20mm, right=18mm, bottom=15mm, footskip=10mm]{geometry}
\usepackage{indentfirst}
\usepackage{amsmath,amssymb}
\usepackage[italicdiff]{physics}
\usepackage{graphicx}
\graphicspath{{images/}}
\DeclareGraphicsExtensions{.pdf,.png,.jpg}
\usepackage{wrapfig}
\usepackage{pgfplots}

\usepackage{caption}
\captionsetup[figure]{name=Рисунок}
\captionsetup[table]{name=Таблица}


\title{\underline{Лабораторная работы 1.2.1}}
\author{Старостин Александр, Б01-401}
\date {26 Ноября, 2024 год}


\begin{document}

\maketitle
\newpage

\textbf{Определение скорости полета пули при помощи баллистического маятника}

\section{Аннотация}
    \par \textbf{Цель работы:} Определить скорость полёта пули применяя законы сохранения и использую баллистические маятники.\\

    \par \textbf{В работе используются:} Духовое ружьё на штативе, осветитель, оптическая система для измерения отклонений маятника, измерительная линейка, пули и весы для их взвешивания, баллистические маятники.


\section{Теоретические сведения}

\subsection{Поступательное движение баллистического маятника}
При контакте пули с цилиндром можно записать ЗСИ:
	\begin{equation}
		 mu = (M+m)V
	\end{equation}
	где $m$ -- масса пули, $u$ -- скорость пули перед ударом, $V$-скорость цилиндра вместе с пулей после удара.
	\begin{equation}
		u=\frac{M+m}{m}V \approx \frac{M}{m}V \;\;\;\;\; V^2=2gh \;\;\;\;\; h = L(1-cos \varphi ) = 2L^2 sin \frac{\varphi^2}{2} \;\;\;\;\;\;\; \varphi \approx \frac{\Delta x}{L}
	\end{equation}
        где $\varphi$ - угол отклонения маятника от вертикали, $\Delta x$ - отклонение маятника\\
	Тогда скорость пули можно выразить как
	\begin{equation} \label{vel1}
	 u=\frac{M}{m} \sqrt{\frac{g}{L}} \Delta x
	\end{equation}

\subsection{Крутильный баллистический маятник}
Считая удар неупругим, можно записать уравнение
	$$mur=I \Omega$$
	$r-$расстояние от линии полёта пули до оси вращения, $I$ -- момент инерции относительно этой оси, $\Omega$ -- угловая скорость маятника сразу после удара.

	Можно пренебречь затуханием колебаний и потерями энергии и записать ЗСЭ:
	$$ k \frac{\varphi^2}{2} = I \frac{\Omega^2}{2} $$
	\noindent где $k$ -- модуль кручения проволоки, $\varphi$ -- максимальный угол поворота маятника, тогда:
	\begin{equation} \label{vel2}
		 u = \varphi \frac{\sqrt{kI}}{mr}
	\end{equation}
\begin{equation}
		\label{phi}
		\varphi \approx \frac{x}{d}
	\end{equation}

	где $x$ -- смещение изображения нити осветителя на шкале.

	Периоды колебаний маятника с грузами и без можно выразить как
	$$T_1= 2 \pi \sqrt{\frac{I - 2MR^2}{k}} \;\;\;\;\;\; T_2 = 2 \pi \sqrt{\frac{I}{k}}$$
	Тогда $\sqrt{kI}$ можно найти как:
	\begin{equation}
		\sqrt{kI} = \frac{4 \pi M R^2 T_2}{T_2^2 - T_1^2}
		\label{kl}
	\end{equation}
	$R$ -- расстояние от оси вращения до центров грузиков, $M$ - масса грузиков.

\section{Ход работы}

\subsection{Поступательное движение баллистического маятника}

\subsubsection{Знакомство с устройством установки}

Мы ознакомились с устройством баллистического маятника и измерительной установки, научились пользоваться духовым ружьём.

\subsubsection{Измерение масс пуль}

Измерим массу каждой пули с помощью точных весов:

\begin{table}[h!]
\centering
\caption{Массы пулей}
\begin{tabular}{|c|c|}
\hline
Номер пули & Масса пули $m \pm{0.005}\text{ г}$  \\ \hline
1  & $0.512$ \\ \hline
2  & $0.512$ \\ \hline
3  & $0.511$ \\ \hline
4  & $0.510$ \\ \hline
5  & $0.513$ \\ \hline
\end{tabular}
\end{table}

\item Масса маятника $M = 2925\pm{5}\text{, г}$

\subsubsection{Измерение высоты подвеса баллистического маятника}

Измерим высоту подвеса баллистического маятника $L$:

\begin{table}[h!]
\centering
\caption{Измерение расстояния $L$}
\begin{tabular}{|c|c|}
\hline
Номер измерения & $L \pm{0.1}\text{ см}$  \\ \hline
1  & $221.4$ \\ \hline
2  & $219.9$ \\ \hline
3  & $222.1$ \\ \hline
4  & $221.1$ \\ \hline
\end{tabular}
\end{table}

\item Среднее расстояние $\overline{L} = 221.1 \pm{0.2} \text{ см}$

\subsubsection{Установка оптической системы}

Мы собрали оптическую систему, предназначенную для измерения перемещения маятника, включили осветитель и добились чёткого изображения шкалы на экране.

\subsubsection{Произведение холостых выстрелов}

Мы произвили несколько холостых выстрелов по маятнику и убедились в том, что он практически не реагирует на удар воздушной струи из ружья.

\subsubsection{Проверка малости затухания колебания}

Мы убедились в малом затухании колебаний: за десять колебаний колебаний их амплитуда уменьшалась меньше, чем наполовину. Ниже приведены результаты измерений:

\begin{table}[h!]
\centering
\caption{Проверка малости затухания колебания}
\begin{tabular}{|c|c|c|}
\hline
Количество колебаний & Амплитуда в начале $A_1\text{, см}$  & Амплитуда в конце $A_2\text{, см}$ \\ \hline
10  & $2.1\pm {0.1}$ & $1.9\pm {0.1}$ \\ \hline

\end{tabular}
\end{table}

\item $\frac{A_1}{A_2} = \frac{2.1}{1.9} = 1.105 < 2$, значит затухание колебаний малое.

\subsubsection{Произведение выстрелов}

Мы произвели 5 выстрелов и определили по формуле (3) скорость пули при каждом выстреле. Результаты измерений приведены в таблице:

\newpage

\begin{table}[h!]
\centering
\caption{Результаты выстрелов}
\begin{tabular}{|c|c|c|c|}
\hline
Номер пули & Отклонение до выстрела  $x_1\text{, мм}$ & Отклонение после выстрела  $x_2\text{, мм}$ & Скорость полёта пули $u\text{, м/с}$\\ \hline
1  & $0.1\pm {0.1}$ & $11.8\pm {0.1}$ & $142.13\pm {2.21}$ \\ \hline
2  & $0.1\pm {0.1}$ & $11.9\pm {0.1}$ & $142.79\pm {2.21}$ \\ \hline
3  & $0.1\pm {0.1}$ & $11.8\pm {0.1}$ & $144.97\pm {2.27}$ \\ \hline
4  & $0.1\pm {0.1}$ & $11.3\pm {0.1}$ & $135.01\pm {2.15}$ \\ \hline
5  & $0.1\pm {0.1}$ & $11.8\pm {0.1}$ & $141.31\pm {2.19}$ \\ \hline

\end{tabular}
\end{table}

\item Приборная погрешность скорости полёта пули $\sigma_u^{\text{приб}} = u \sqrt{(\frac{\sigma_M}{M})^2 + (\frac{\sigma_m}{m})^2 + (\frac{\sigma_L}{2L})^2 + (\frac{\sqrt{\sigma_{x_1}^2 + \sigma_{x_2}^2}}{x_2 - x_1})^2}$

\subsubsection{Определение погрешности скорости пули в каждом выстреле}

Погрешность скорости каждой пули при выстреле расчитаны в предыдущем пункте.

\subsubsection{Определение средней скорости полёта пули}

Из таблицы получаем, что:

\item Средняя скорость полёта пули $\overline{u} = 141.24 \pm{2.77} \text{, м/с}$

\item Погрешность измерения средней скорости полёта пули $\sigma_u = \sqrt{(\sigma_u^{\text{приб}})^2 + (\sigma_u^{\text{случ}})^2}$

\item $\sigma_u^{\text{приб}}$ смотреть в пункте 3.1.7

\item $\sigma_u^{\text{случ}} = \sqrt{\frac{1}{N(N-1)} \sum_{i = 1}^{N} (u_i - \overline{u})^2}$

\item $N = 5$


\newpage

\subsection{Крутильный баллистический маятник}

\subsubsection{Знакомство с устройством установки}

Мы ознакомились с конструкцией установки и научились пользоваться духовым ружьём.

\subsubsection{Измерение масс пуль}

Измерим массу каждой пули с помощью точных весов:

\begin{table}[h!]
\centering
\caption{Массы пулей}
\begin{tabular}{|c|c|}
\hline
Номер пули & Масса пули $m \pm{0.005}\text{ г}$  \\ \hline
1  & $0.512$ \\ \hline
2  & $0.514$ \\ \hline
3  & $0.502$ \\ \hline
4  & $0.516$ \\ \hline
5  & $0.515$ \\ \hline
\end{tabular}
\end{table}

\item Масса каждого груза $M = 735.5 \pm{0.3} \text{, г}$

\subsubsection{Измерение размеров установки}

Измерим размеры установки:

\item Расстояние от оси вращения до центра масс одного из грузов $R = 33.5 \pm{0.1} \text{ см}$

\item Расстояние от оси вращения до центра одной из мишеней $r = 22.0 \pm{0.1} \text{ см}$

\item Расстояние от оси вращения до шкалы $d = 137.0 \pm{0.2} \text{ см}$

\subsubsection{Установка оптической системы}

Мы собрали оптическую систему, предназначенную для измерения поворота маятника, включили лазер и добились его отметки на нуле шкалы.

\subsubsection{Произведение холостых выстрелов}

Мы произвили несколько холостых выстрелов по маятнику и убедились в том, что он практически не реагирует на удар воздушной струи из ружья.

\subsubsection{Проверка малости затухания колебания}

Мы убедились в малом затухании колебаний: за десять колебаний колебаний их амплитуда уменьшалась меньше, чем наполовину. Ниже приведены результаты измерений:

\begin{table}[h!]
\centering
\caption{Проверка малости затухания колебания}
\begin{tabular}{|c|c|c|}
\hline
Количество колебаний & Амплитуда в начале $A_1\text{, см}$  & Амплитуда в конце $A_2\text{, см}$ \\ \hline
10  & $12.1\pm {0.1}$ & $10.6\pm {0.1}$ \\ \hline

\end{tabular}
\end{table}

\item $\frac{A_1}{A_2} = \frac{12.1}{10.6} = 1.142 < 2$, значит затухание колебаний малое.

\subsubsection{Измерение периодов}

Результаты измерений периодов приведены в таблице:

\begin{table}[h!]
\centering
\caption{Измерение периодов колебаний}
\begin{tabular}{|c|c|c|}
\hline
Количество колебаний & Время с грузами $t_1\text{, с}$  & Время без грузов $t_2\text{, с}$ \\ \hline
10  & $67.0\pm {0.3}$ & - \\ \hline
10  & $67.1\pm {0.3}$ & - \\ \hline
10  & $67.1\pm {0.3}$ & - \\ \hline
10  & - & $50.1\pm {0.3}$ \\ \hline
10  & - & $50.2\pm {0.3}$ \\ \hline
10  & - & $50.1\pm {0.3}$ \\ \hline


\end{tabular}
\end{table}

\item Средний период с грузами $\overline{T_1} = 6.71 \pm{0.03} \text{, с}$

\item Средний период без грузов $\overline{T_2} = 5.01 \pm{0.03} \text{, с}$

\item Погрешность периода $\sigma_T = \sqrt{(\sigma_T^{\text{приб}})^2 + (\sigma_T^{\text{случ}})^2}$

\item $\sigma_T^{\text{приб}} = T \frac{\sigma_t}{t}$

\item $\sigma_T^{\text{случ}} = \sqrt{\frac{1}{N(N-1)} \sum_{i = 1}^{N} (T_i - \overline{T})^2}$

\item $N = 3$

По формуле (6) найдём:

\item $\sqrt{kI} = 0.35 \pm{0.1} \text{, м$^2$ * кг/c$^2$}$

\item Погрешность измеренной величины $\sigma_{\sqrt{kI}} = \sqrt{kI} \sqrt{(\frac{\sigma_M}{M})^2 + (\frac{4\sigma_R}{R})^2 + (\frac{\sigma_{T_2}}{T_2})^2 + (\frac{\sqrt{(2\sigma_{T_2})^2+(2\sigma_{T_1})^2}}{T_2^2 - T_1^2})^2}$



\subsubsection{Произведение выстрелов}

Мы произвели 5 выстрелов и определили по формуле (4) скорость пули при каждом выстреле. Результаты измерений приведены в таблице:

\begin{table}[h!]
\centering
\caption{Результаты выстрелов}
\begin{tabular}{|c|c|c|c|}
\hline
Номер пули & Отклонение до выстрела  $x_1\text{, мм}$ & Отклонение после выстрела  $x_2\text{, мм}$ & Скорость полёта пули $u\text{, м/с}$\\ \hline
1  & $0.9\pm {0.1}$ & $4.3\pm {0.1}$ & $77.19\pm {11.74}$ \\ \hline
2  & $0.3\pm {0.1}$ & $4.1\pm {0.1}$ & $86.27\pm {12.98}$ \\ \hline
3  & $0.2\pm {0.1}$ & $4.4\pm {0.1}$ & $95.54\pm {14.25}$ \\ \hline
4  & $0.2\pm {0.1}$ & $4.0\pm {0.1}$ & $86.61\pm {13.03}$ \\ \hline
5  & $0.1\pm {0.1}$ & $4.2\pm {0.1}$ & $92.90\pm {13.85}$ \\ \hline

\end{tabular}
\end{table}

\item Приборная погрешность скорости полёта пули $\sigma_u^{\text{приб}} = u \sqrt{(\frac{\sigma_{\sqrt{kI}}}{2\sqrt{kI}})^2 + (\frac{\sigma_m}{m})^2 + (\frac{\sigma_r}{r})^2 + (\frac{\sigma_d}{d})^2 + (\frac{\sqrt{\sigma_{x_1}^2 + \sigma_{x_2}^2}}{x_2 - x_1})^2}$

\subsubsection{Определение погрешности скорости пули в каждом выстреле}

Погрешность скорости каждой пули при выстреле расчитаны в предыдущем пункте.

\subsubsection{Определение средней скорости полёта пули}

Из таблицы получаем, что:

\item Средняя скорость полёта пули $\overline{u} = 87.70 \pm{13.55} \text{, м/с}$

\item Погрешность измерения средней скорости полёта пули $\sigma_u = \sqrt{(\sigma_u^{\text{приб}})^2 + (\sigma_u^{\text{случ}})^2}$

\item $\sigma_u^{\text{приб}}$ смотреть в пункте 3.2.8

\item $\sigma_u^{\text{случ}} = \sqrt{\frac{1}{N(N-1)} \sum_{i = 1}^{N} (u_i - \overline{u})^2}$

\item $N = 5$

\section{Вывод}

Мы определили средние скорости полёта пулей из разных ружей, применяя законы сохранения и используя баллистические маятники.

\end{document}





